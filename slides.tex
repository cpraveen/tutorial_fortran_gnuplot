% $Header: /cvsroot/latex-beamer/latex-beamer/solutions/conference-talks/conference-ornate-20min.en.tex,v 1.6 2004/10/07 20:53:08 tantau Exp $

\documentclass{beamer}

\mode<presentation>
{
  %\usetheme{Madrid}
  % or ...

  \setbeamercovered{transparent}
  % or whatever (possibly just delete it)
}


%\usepackage{graphicx}
%\usepackage{aliases}
\usepackage{verbatim}

\usepackage[english]{babel}
% or whatever

%\usepackage[latin1]{inputenc}
% or whatever

\usepackage{times}
%\usepackage[T1]{fontenc}
% Or whatever. Note that the encoding and the font should match. If T1
% does not look nice, try deleting the line with the fontenc.


\title[Tutorial on computing] % (optional, use only with long paper titles)
{Tutorial on computing}

%\subtitle
%{Include Only If Paper Has a Subtitle}

\author[] % (optional, use only with lots of authors)
{Praveen. C}
% - Give the names in the same order as the appear in the paper.
% - Use the \inst{?} command only if the authors have different
%   affiliation.

\institute[TIFR-CAM] % (optional, but mostly needed)
{
   TIFR-CAM \\
   Bangalore \\
{\tt praveen@math.tifrbng.res.in} \\
}
% - Use the \inst command only if there are several affiliations.
% - Keep it simple, no one is interested in your street address.

\date[19 March, 2008] % (optional, should be abbreviation of conference name)
{19 March, 2008}
% - Either use conference name or its abbreviation.
% - Not really informative to the audience, more for people (including
%   yourself) who are reading the slides online

\subject{Computing}
% This is only inserted into the PDF information catalog. Can be left
% out. 



% If you have a file called "university-logo-filename.xxx", where xxx
% is a graphic format that can be processed by latex or pdflatex,
% resp., then you can add a logo as follows:

% \pgfdeclareimage[height=0.5cm]{university-logo}{university-logo-filename}
% \logo{\pgfuseimage{university-logo}}

% Delete this, if you do not want the table of contents to pop up at
% the beginning of each subsection:
%\AtBeginSubsection[]
%{
%  \begin{frame}<beamer>
%    \frametitle{Outline}
%    \tableofcontents[currentsection,currentsubsection]
%  \end{frame}
%}

% If you wish to uncover everything in a step-wise fashion, uncomment
% the following command: 

%\beamerdefaultoverlayspecification{<+->}

\begin{document}

\begin{frame}
  \titlepage
\end{frame}

\begin{frame}
  \frametitle{Outline}
  \tableofcontents
  % You might wish to add the option [pausesections]
\end{frame}
%#############################################################################
\section{Platform and tools}
%-----------------------------------------------------------------------------
\begin{frame}[fragile]
  \frametitle{Platform and tools}

  \begin{itemize}

  \item We will use a UNIX or Linux-style OS (includes MACOSX)
  \item Command prompt denoted with a dollar sign
  \begin{verbatim}
  $
  \end{verbatim}
  \item C compilers

  \begin{itemize}
  \item GNU: gcc/cc, g++
  \item Intel: icc
  \item PGI: pgcc
  \end{itemize}

  \item Fortran compilers

  \begin{itemize}
  \item GNU: g77 (old), gfortran
  \item Intel: ifort
  \item PGI: pgf77, pgf90, pgf95
  \end{itemize}

  \end{itemize}

\end{frame}
%-----------------------------------------------------------------------------
\begin{frame}[fragile]
  \frametitle{Platform and tools}

  \begin{itemize}

  \item Check if the compiler is installed

  \item C compiler

  \begin{verbatim}
  $ cc
  gcc: no input files
  \end{verbatim}

  \item Fortran compiler

  \begin{verbatim}
  $ gfortran
  gfortran: no input files
  \end{verbatim}

  \end{itemize}

\end{frame}
%-----------------------------------------------------------------------------
\end{document}
